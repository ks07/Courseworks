\documentclass[paper=a4, fontsize=12pt]{article}
\usepackage[utf8]{inputenc}
\usepackage{amsmath}

\begin{document}

\section{Question 4}

\subsection{4.i. 6 Approximation}

A 3 approximation for this bin packing problem can be found by making use of
the FirstFitDecreasing algorithm, with the set of all gems partitioned into two
subsets. One such subset should hold only a single type of gem, whilst the
other subset should hold the remaining two types of gem. The following proofs
do not consider the effect of choosing different gem types to partition around;
in my algorithm I will simply partition the n input gems into a subset AB,
containing all gems of type A and B, and a subset C, containing all gems of
type C. The following pseudocode makes use of a FirstFitDecreasing function,
that implements the algorithm as discussed in the lecture slides. This function
takes as input a list of gems and outputs the number of sacks required.

\begin{verbatim}
# Receive as input a list N of n gems, i.e. n = |N|

AB = []
C  = []

# Partition the gems, O(n)
for gem in N:
  if gem.type == C:
    C.append(gem)
  else:
    AB.append(gem)

s = 0

# As only two types of gem, cannot mix more than two types in a sack. O(n^2)
s = FirstFitDecreasing(AB)

# Repeat for type C, again O(n^2)
s = s + FirstFitDecreasing(C)

# s should now hold the count of bins required
return s
\end{verbatim}

\subsubsection{4.i.a Runtime Analysis}
From the lecture slides, week 13, we have that the FirstFitDecreasing algorithm
is \(O(n^2)\), from the sorting operation that needs to be performed before
assigning items to bins. As the annotated pseudocode describes, we perform an
\(O(n)\) preprocessing step whereby the gems are split into two subsets. This
is a simple loop over all elements in N; this loop is executed n times,
performing \(O(1)\) operations in the body of the loop. The algorithm then runs
the FirstFitDecreasing algorithm twice, giving a worst case runtime of \(O(2n^2
+ n)\), which is equivalent to \(O(n^2)\). By definition, this is polynomial
time.

\subsubsection{4.i.b Correctness}
From the lecture slides, week 13, the FirstFitDecreasing algorithm produces
valid solutions; the algorithm operates by finding the first bin that satisfies
the equation \(t + w <= c\), where t is the total weight of items already in
the bin, w is the weight of the item currently being considered, and c is the
capacity of each bin (c = 1.0, in this case).

The algorithm must therefore produce valid solutions for both subsets of N (AB
and C), at least in terms of sack capacity. However, the problem also specifies
a restriction on the mixing of gem types; no sack may contain a gem of type A,
a gem of type B, and a gem of type C. This algorithm avoids violating this
condition by partitioning the gems such that the FirstFitDecreasing algorithm
is never supplied with gems of all three types at once, so it cannot produce an
invalid packing. The algorithm combines the two packings (of AB and of C), by
simply concatenating the two; or in this case, by summing the outputs of
FirstFitDecreasing. In other words, the number of sacks needed for all gems of
type A and B is found, then the number for type C is found, and the final
solution keeps type C completely separated from the others.

\subsubsection{4.i.c Approximation Factor}
From the lecture slides, week 13, we are shown that the FirstFitDecreasing
algorithm produces a \(3/2\) approximation. \((Opt <= s <= (3/2) *
Opt)\). Consider each subset individually; denoting the optimal solution for
the AB subset as \(Opt_{AB}\), and similarly \(Opt_C\) for the C subset. Running FFD on
these subsets will produce an \(s_x (x = AB | C)\) that satisfies: \(Opt_{AB} <=
s_{AB} <= (3/2) * Opt_{AB}\) and \(Opt_C <= s_C <= (3/2) * Opt_C\).

It must be that \(Opt_C <= Opt\), where Opt is the global optimum. This can be
explained by considering that C is a subset of N, thus by definition C must
have at most n elements. Therefore, in the most extreme case the entirety of N
will in fact be gems of type C - thus the local optimum packing for C will be
equal to the global optimum for N, Opt. The only other possibility is that the
subset C has fewer elements than N. Clearly in this case, \(Opt_C <= Opt\), as
removing items from the input without modifying the weights of other items
cannot possibly increase the required number of bins. It is important to note
that all gems have positive weights.

The same argument can be made for the subset AB, as there are no restrictions
on sacks containing only two types of gem other than the standard capacity
limit. Thus it follows that \(Opt_{AB} <= Opt\).

To summarise, it can be shown that we now have two values, \(s_AB\) and
\(s_C\), that both satisfy the inequality \(Opt <= s_x <= (3/2) * Opt\). The
algorithm described here produces an output s, where \(s = s_{AB} +
s_C\). Combining this value of s with the approximation factors shown earlier
gives us inequalities as such:

\begin{alignat*}{2}
Opt_{AB} + Opt_C &\leq s_{AB} &+ s_C &\leq \frac{3}{2} Opt_{AB} + \frac{3}{2} Opt_C\\
Opt + Opt &\leq &s \quad &\leq \frac{3}{2} Opt + \frac{3}{2} Opt\\
2 Opt &\leq &s \quad &\leq 2 (\frac{3}{2} Opt)\\
2 Opt &\leq &s \quad &\leq{} 3 Opt
\end{alignat*}

\end{document}
